%        File: slide1.tex
%     Created: 六  6 02 02:00 下午 2018 C
% Last Change: 六  6 02 02:00 下午 2018 C
%
\documentclass[UTF8, xcolor=table, compress]{beamer}
%\usepackage{fontspec}
%\setsansfont{宋体}
\usepackage[BoldFont,SlantFont]{xeCJK}
%\setCJKmainfont[BoldFont={Adobe Heiti Std},ItalicFont={Adobe Kaiti Std}]{SimSun}

\usepackage{latexsym,amssymb,amsmath,amsbsy,amsopn,amstext,xcolor,multicol}
\usepackage{graphicx,wrapfig,fancybox}
\usepackage{pgf,pgfarrows,pgfnodes,pgfautomata,pgfheaps,pgfshade}
\usepackage{$HOME/.vim/ftplugin/latex-suite/packages/shenlan}
%\usepackage[backend=bibtex,style=IEEE,sorting=none]{biblatex} % [参考文献格式](https://www.sharelatex.com/blog/2013/07/31/getting-started-with-biblatex.html)
\usepackage[backend=bibtex,style=authoryear-comp, sorting=none]{biblatex} % [参考文献格式](https://www.sharelatex.com/blog/2013/07/31/getting-started-with-biblatex.html) %mac IEEE not found
\usepackage{array}
\usepackage{bm}
\usepackage{caption}
\usepackage[caption=false,font=scriptsize]{subfig}
\usepackage{multirow}
\usepackage{booktabs}
\usepackage{tikz}
\usepackage{tikzscale}
\usepackage{animate}
%\usepackage{times} %与上面的冲突,加上这个 粗体斜体就失效
%\usepackage{mathptmx}


\defbibheading{bibliography}[\bibname]{} %avoid printbibliography 自动生成目录
\addbibresource{ref/papers-bib-in-google.bib}
\addbibresource{ref/chinese-ref.bib}
%\setbeamertemplate{bibliography item}{\insertbiblabel} %将列表中默认的丑陋的icon 改成数字,或者下面这个也行
\setbeamertemplate{bibliography item}[text] % [ref](http://tex.stackexchange.com/questions/68080/beamer-bibliography-icon)
%\setbeamertemplate{footline}[frame number]{}

%\setframeofframes{of}

\usepackage{boxedminipage} %for: bvh border
\def\fourgraphicswidth{0.35} %0.3\textwidth


\begin{document}

\setbeamerfont{footnote}{size=\tiny}
\setbeamerfont{caption}{size=\scriptsize}
\setbeamertemplate{caption}[numbered]
\setbeamerfont{subsection in toc}{size=\footnotesize}
\renewcommand*{\bibfont}{\footnotesize}

\title{<+ +>}
\author[陈达贵]{\hskip 3em\includegraphics[width=0.5\linewidth]{$HOME/Pictures/Photo/shenlan-avatar.png}}

\institute[深蓝学院~机器学习~\& 强化学习理论与实践]{\small \vskip 38pt强化学习理论与实践}
%\date{2015-06-07}
%\date{\small \vskip -17pt二〇一五年六月}
\date{\today}



\begin{frame}
\vspace{-15mm}
\titlepage{}
\vspace{-50mm}
\begin{figure}[htbp]
\begin{center}
\includegraphics[width=0.3\linewidth]{$HOME/Pictures/Photo/shenlan-whole.png}
\end{center}
\end{figure}
%\beign{picture}(1,1)
%\put(6,8){\includegraphics[width=0.15\linewidth]{Tsinghua_University_Logo.eps}}
%\end{picture}

\end{frame}


\section*{目录}
\begin{frame}[shrink]
\frametitle{\secname}
% \begin{multicols}{2}  % 如果目录太多,可以使用双栏显示
\tableofcontents[hidesubsections]
% \tableofcontents[sections={<1-5>}]
%\end{multicols}
\note{这里是注释}
\end{frame}

\AtBeginSection[] {
\begin{frame}[shrink]
\frametitle{\includegraphics[height=1em]{$HOME/Pictures/Photo/shenlan-logo.png} \quad 目录}
\tableofcontents[sectionstyle=show/shaded,subsectionstyle=show/shaded/hide]
%\tableofcontents[current,currentsubsection,hideothersubsections]
% \tableofcontents[current,currentsubsection,sections={<1-5>}]
\end{frame}
\addtocounter{framenumber}{-1}  %目录页不计算页码
}
\AtBeginSubsection[] {
\begin{frame}[shrink]
\frametitle{\includegraphics[height=1em]{$HOME/Pictures/Photo/shenlan-logo.png} \quad 目录}
\tableofcontents[sectionstyle=show/shaded,subsectionstyle=show/shaded/hide]
%\tableofcontents[current,currentsubsection,hideothersubsections]
% \tableofcontents[current,currentsubsection,sections={<1-5>}]
\end{frame}
\addtocounter{framenumber}{-1}  %目录页不计算页码
}

\section{Section I}

\subsection{Subsection I}

\frame{
%\frametitle{\subsecname~}
Itemize item, just items:
\begin{itemize}[<+-| alert@+>]
\item
item a
\item
item b
\item
item c
\item
item d
\end{itemize}
}

\subsection{Subsection II }
\frame{
%\frametitle{\subsecname~Goods}
The enumarate item:
\begin{enumerate}[<+-| alert@+>]
\item
Item start with number
\item
This is item 2
\end{enumerate}
}

\section{section II}

\subsection{subsection I}

\frame{
Description item:
\begin{description}%[<+-| alert@+>]
\item[description] Item starts with description.
\item[Fermion] This is item 2.
\end{description}
}

\section{Discussion}
\subsection{总结}
\subsection{致谢}
\frame{
\begin{columns}
\column{3cm}
\column{4cm}
Thank you!
\column{3cm}
\end{columns}
}
\end{document}
